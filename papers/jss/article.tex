\documentclass[article]{jss}

%%%%%%%%%%%%%%%%%%%%%%%%%%%%%%
%% declarations for jss.cls %%%%%%%%%%%%%%%%%%%%%%%%%%%%%%%%%%%%%%%%%%
%%%%%%%%%%%%%%%%%%%%%%%%%%%%%%

%% almost as usual
\author{Ingmar Visser\\University of Amsterdam \And 
        Maarten Speekenbrink\\University College London }
\title{\pkg{depmixS4}: An \proglang{R}-package for hidden Markov models}

%% for pretty printing and a nice hypersummary also set:
\Plainauthor{Ingmar Visser, Maarten Speekenbrink} %% comma-separated
\Plaintitle{depmixS4: An R-package for hidden Markov models} %% without formatting
\Shorttitle{depmixS4 Hidden Markov Models} %% a short title (if necessary)

%% an abstract and keywords
\Abstract{
	\pkg{depmixS4} implements a general framework for definining and
	fitting hidden Markov mixture model in the R programming language
	\citep{R2009}.  This includes standard Markov models,
	latent/hidden Markov models, and latent class and finite mixture
	distribution models.  The models can be fitted on mixed
	multivariate data with multinomial and/or gaussian distributions.
	Parameters can be estimated subject to general linear constraints.
	Parameter estimation is done through an EM algorithm or by a
	direct optimization approach using the
	\pkg{Rdonlp2} optimization routine when contraints are imposed on
	the parameters.  A number of illustrative examples are included.
	
}
\Keywords{hidden Markov model, dependent mixture model, mixture model, \proglang{R}}
\Plainkeywords{hidden Markov model, dependent mixture model, mixture model,R} %% without formatting
%% at least one keyword must be supplied

%% publication information
%% NOTE: Typically, this can be left commented and will be filled out by the technical editor
%% \Volume{13}
%% \Issue{9}
%% \Month{September}
%% \Year{2004}
%% \Submitdate{2004-09-29}
%% \Acceptdate{2004-09-29}

%% The address of (at least) one author should be given
%% in the following format:
\Address{
Ingmar Visser\\
Department of Psychology\\
University of Amsterdam,
Roetersstraat 15\\
1018 WB, Amsterdam\\
The Netherlands\\
E-mail: \email{i.visser@uva.nl} \\
URL: \url{http://www.ingmar.org/}
}
%% It is also possible to add a telephone and fax number
%% before the e-mail in the following format:
%% Telephone: +43/1/31336-5053
%% Fax: +43/1/31336-734

%% for those who use Sweave please include the following line (with % symbols):
%% need no \usepackage{Sweave.sty}

%% end of declarations %%%%%%%%%%%%%%%%%%%%%%%%%%%%%%%%%%%%%%%%%%%%%%%


\begin{document}

%% include your article here, just as usual
%% Note that you should use the \pkg{}, \proglang{} and \code{} commands.

\section[]{Introduction}
%% Note: If there is markup in \(sub)section, then it has to be escape as above.

Why did we develop depmixS4?

\section{The dependent mixture model}

Describe the model formulae

\subsection{Likelihood}

Give some refs to computing the likelihood

\subsection{Parameter estimation}

Give some refs to EM/forward/backbward and possibly others.


\section{Using \pkg{depmixS4}}

Explain the three steps involved in fitting a model. 


\subsection{Start-up example}

Just the RT data.


\subsection{Adding covariates on transition parameters}


Give an example on how to do this.


\subsection{Add covariates to prior model}

Give an example with the balance data. 

\section{Extending \pkg{depmixS4}}

Introduce the exgaus example. 

Explain the use of makeDepmix for having full control of every aspect of the model. 


\section{Conclusion \& future work}


What are our next plans?


\bibliography{all,ingmar}


\end{document}
