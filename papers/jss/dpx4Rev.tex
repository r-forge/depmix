% -*- mode: noweb; noweb-default-code-mode: R-mode; -*-

\documentclass[article]{jss}
\usepackage{amsmath}

%\usepackage[]{amsmath, amsfonts, amstext, amsthm} 
%\usepackage{amssymb}
%\usepackage[]{graphics} 
%\usepackage{epsfig}
%\usepackage{epstopdf}

%%%%%%%%%%%%%%%%%%%%%%%%%%%%%%
%% declarations for jss.cls %%%%%%%%%%%%%%%%%%%%%%%%%%%%%%%%%%%%%%%%%%
%%%%%%%%%%%%%%%%%%%%%%%%%%%%%%

%% almost as usual
\author{Ingmar Visser\\University of Amsterdam \And 
        Maarten Speekenbrink\\University College London}
        
\title{\pkg{depmixS4} : An \proglang{R}-package for hidden Markov models}


%% for pretty printing and a nice hypersummary also set:
\Plainauthor{Ingmar Visser, Maarten Speekenbrink} %% comma-separated
\Plaintitle{depmixS4: An R-package for hidden Markov models} %% without formatting

\Shorttitle{depmixS4: Hidden Markov Models} %% a short title (if necessary)

%% an abstract and keywords
\Abstract{
	
	\pkg{depmixS4} implements a general framework for defining and
	estimating dependent mixture models in the \proglang{R}
	programming language \citep{R2009}.  This includes standard Markov
	models, latent/hidden Markov models, and latent class and finite
	mixture distribution models.  The models can be fitted on mixed
	multivariate data with distributions from the \code{glm} family,
	the logistic multinomial, or the multivariate normal distribution.
	Other distributions can be added easily, and an example is
	provided with the exgaus distribution.  Parameters are estimated by
	the EM algorithm or, when (linear) constraints are imposed on the
	parameters, by direct numerical optimization with the
	\pkg{Rdonlp2} routine.}

\Keywords{hidden Markov model, dependent mixture model, mixture model}

\Plainkeywords{hidden Markov model, dependent mixture model, mixture model} %% without formatting
%% at least one keyword must be supplied

%% publication information
%% NOTE: Typically, this can be left commented and will be filled out by the technical editor
%% \Volume{13}
%% \Issue{9}
%% \Month{September}
%% \Year{2004}
%% \Submitdate{2004-09-29}
%% \Acceptdate{2004-09-29}

%% The address of (at least) one author should be given
%% in the following format:
\Address{
Ingmar Visser\\
Department of Psychology\\
University of Amsterdam\\
Roetersstraat 15\\
1018 WB, Amsterdam\\
The Netherlands\\
E-mail: \email{i.visser@uva.nl} \\
URL: \url{http://www.ingmar.org/}
}

%% It is also possible to add a telephone and fax number
%% before the e-mail in the following format:
%% Telephone: +43/1/31336-5053
%% Fax: +43/1/31336-734

\newcommand{\vc}{\mathbf}
\newcommand{\mat}{\mathbf}
\newcommand{\greekv}[1]{\mbox{\boldmath$\mathrm{#1}$}}
\newcommand{\greekm}[1]{\mbox{\boldmath$#1$}}

%% end of declarations %%%%%%%%%%%%%%%%%%%%%%%%%%%%%%%%%%%%%%%%%%%%%%%

%\batchmode




\usepackage{a4wide}

% \batchmode

\usepackage{Sweave}
\begin{document}

\maketitle

%% include your article here, just as usual
%% Note that you should use the \pkg{}, \proglang{} and \code{} commands.

% Refs to check: Jansen 2002, Dutilh 2009

\begin{center}
\bf{DRAFT: DO NOT QUOTE WITHOUT CONTACTING AUTHOR}
\end{center}


A simple example that will run in any S engine: The integers from 1 to
10 are
\begin{Schunk}
\begin{Soutput}
 [1]  1  2  3  4  5  6  7  8  9 10
\end{Soutput}
\end{Schunk}

We can also emulate a simple calculator:
\begin{Schunk}
\begin{Sinput}
> 1 + 1
\end{Sinput}
\begin{Soutput}
[1] 2
\end{Soutput}
\begin{Sinput}
> 1 + pi
\end{Sinput}
\begin{Soutput}
[1] 4.141593
\end{Soutput}
\begin{Sinput}
> sin(pi/2)
\end{Sinput}
\begin{Soutput}
[1] 1
\end{Soutput}
\end{Schunk}

Now we look at Gaussian data:

\begin{Schunk}
\begin{Soutput}
 [1]  0.36039435 -1.83775608  1.04278935 -1.07035396 -0.82150164 -0.32257866
 [7]  0.97533933 -0.97446747  1.25840051 -1.01986278  1.15659229 -0.71535551
[13]  0.76128983  0.35511419  0.30919687 -1.45434289 -0.09874472 -0.13624640
[19] -1.57319360  1.32749121
\end{Soutput}
\begin{Soutput}
	One Sample t-test

data:  x 
t = -0.5432, df = 19, p-value = 0.5933
alternative hypothesis: true mean is not equal to 0 
95 percent confidence interval:
 -0.6012708  0.3534912 
sample estimates:
 mean of x 
-0.1238898 
\end{Soutput}
\end{Schunk}
Note that we can easily integrate some numbers into standard text: The
third element of vector \texttt{x} is 1.04278935363803, the
$p$-value of the test is 0.59332. % $

Now we look at a summary of the famous iris data set, and we want to
see the commands in the code chunks:



% the following code is R-specific, as data(iris) will not run in Splus. 
% Hence, we mark it as R code. 
\begin{Schunk}
\begin{Sinput}
> data(iris)
> summary(iris)
\end{Sinput}
\begin{Soutput}
  Sepal.Length    Sepal.Width     Petal.Length    Petal.Width   
 Min.   :4.300   Min.   :2.000   Min.   :1.000   Min.   :0.100  
 1st Qu.:5.100   1st Qu.:2.800   1st Qu.:1.600   1st Qu.:0.300  
 Median :5.800   Median :3.000   Median :4.350   Median :1.300  
 Mean   :5.843   Mean   :3.057   Mean   :3.758   Mean   :1.199  
 3rd Qu.:6.400   3rd Qu.:3.300   3rd Qu.:5.100   3rd Qu.:1.800  
 Max.   :7.900   Max.   :4.400   Max.   :6.900   Max.   :2.500  
       Species  
 setosa    :50  
 versicolor:50  
 virginica :50  
\end{Soutput}
\end{Schunk}


\begin{figure}[htbp]
  \begin{center}
\begin{Schunk}
\begin{Sinput}
> library(graphics)
> pairs(iris)
\end{Sinput}
\end{Schunk}
\includegraphics{dpx4Rev-006}
    \caption{Pairs plot of the iris data.}
  \end{center}
\end{figure}

\begin{figure}[htbp]
  \begin{center}
\begin{Schunk}
\begin{Sinput}
> boxplot(Sepal.Length ~ Species, data = iris)
\end{Sinput}
\end{Schunk}
\includegraphics{dpx4Rev-007}
    \caption{Boxplot of sepal length grouped by species.}
  \end{center}
\end{figure}


% R is not S-PLUS, hence this chunk will be ignored:

\end{document}


