%&LaTeX

\documentclass[a4paper,man,nobf]{apa}
 \usepackage[]{amsmath, amsfonts, amstext, amsthm}
 \usepackage[oztex, final]{graphics}
 \usepackage[]{epsfig}

\usepackage[round]{natbib}

\title{A comprehensive approach to fitting multivariate latent Markov 
models and latent transition models: An implementation and 
applications}

\author{Ingmar Visser}
\affiliation{Developmental Processes Research Group, Department of 
Psychology, University of Amsterdam}
\date{\today}
 
% \acknowledgements{Author notes etc.}   
 \rightheader{Depedent Mixtures}                    % defaults to title
\shorttitle{Depedent Mixtures}                      % defaults to title
% \leftheader{Authors}         

%\batchmode
	
%\newcommand{\citep}{\cite}
%\newcommand{\citet}{\citeA}

\newcommand{\vc}{\mathbf}
\newcommand{\mat}{\mathbf}

\abstract{In this article, an outline is presented for a program that 
computes parameter estimates with standard errors for the 
multivariate hidden Markov model. The model itself can be seen as an 
extension and generalization of both the hidden or latent Markov 
model and latent class models. The parameter procedure is different 
from the traditional approach with these models. This approach allows 
modeling of arbitrarily long time series efficiently and without 
exponentially growing computational effort. Moreover, because 
analytical gradients and Hessian are available, models can easily be 
fitted subject to arbitrary linear and non-linear constraints. 
Multi-group and mixture analyses are also possible. A number of 
illustrative examples are presented: 1) mixture analysis of 
(Markovian) learning data , 2) analysis of sleep stages from EEG, and 
3) intra- and inter-individual differences in personality structure.}

%\keywords{latent Markov model; latent class model; latent transition model; mixtures; gradients; 
%%%% Hessian; estimation}

\begin{document}

%\bibliographystyle{plainnat}

\maketitle



In this paper an outline is provided for a program that computes 
parameter estimates for the latent/hidden Markov mode (LMM) and for 
latent transition analysis (LTA), both univariate and multivariate.  
The difference between the two types of model is not a principled one 
but rather of degree, being N is small and T large in LMM and vice 
versa in LTA. Even so, when T is indeed large as is usually the case 
in the hidden Markov model literature, the transition matrix is taken 
to be stationary. The basis for the current approach is to use the 
hidden Markov model (HMM) estimation equations since these are 
suitable for long time series which is not the case for the standard 
EM algorithm used for latent class analysis (LCA) and LTA which soon 
runs into the problem of sparse contingency tables and underflow 
problems during computation.  The approach is illustrated using a 
number of simulated and real data examples.


\section{Introduction}

The model that is considered in this paper is the multivariate (or 
multi indicator) latent Markov model. The main contribution of this 
paper is the extension of estimation algorithms to allow for 
arbitrary length time series of observations. In traditional 
estimation methods this is impossible due to underflow problems. 
Moreover, the estimation procedure that I present is not based on the 
complete contingency table, which is impossible or impractical in the 
case of long time series. In particular, one might want to model a 
single time series. The usual approach to latent Markov models is 
then faced with a huge contingency table, $2^{T}$ cells for a binary 
variable measured at $T$ time points, in which only a single cell has 
a one and the others are zero. Howzthat for sparsity? 

The model can be seen as an extension of several other models, in 
particular, the hidden Markov model (HMM), the latent class model 
(LCM), the latent transition model (LTM) and also the simple Markov 
model (MM). Markov models (and their subsequent refinements) have a 
long history in psychology, mainly in the psychology of learning 
\citep{Wic82}. These models have in common that they are all used to 
describe one or more sequences of responses $O_{1}, \ldots, O_{T}$, 
where each of the $O_{t}$'s may be multivariate. In this paper mixed 
categorical and continuous observations are considered. 

The model concerned in this paper is the combination of the latent 
class model and the hidden Markov model. There is quite some work 
already on this model, but it is usually limited to small numbers of 
repeated meausurements, in which case this model is usually called 
the latent transition model: it is the repeated measurements 
extension of the latent class model. It is customary in those models 
to have time-varying (transition) parameters. In the time series 
literature on the other hand, where hidden Markov models are 
frequently used, transition parameters are taken to be 
time-homogeneous.  This assumption is necessary when analyzing single 
time-series since at each time $t$ only a single transition is 
observed. The multi-variate latent Markov model is defined by the 
following elements:
\begin{enumerate}
  	\item a set $\vc{S}$ of latent states $S_{i},\, i=1, \ldots , n$,
  	\item a matrix $\mat{A}$ of transition probabilities $a_{ij}$ for 
the transition 
	from state $S_{i}$ to state $S_{j}$,
  	\item a set $\vc{B}$ of observation functions $b_{j}(\cdot)$ that 
provide the conditional 	probabilities associated with latent state 
$S_{j}$, 
  	\item a vector $\pmb{\pi}$ of latent state initial probabilities 
$\pi_{i}$.
\end{enumerate}
When transitions are added to the latent class model, it is more 
appropriate to refer to the classes as states. The word class is 
rather more associated with a stable trait-like attribute whereas a 
state can change over time. 

The data has the general form $O_{1}^{1}, \ldots, O_{1}^{m}, 
O_{2}^{1}, \ldots, O_{2}^{m}, \ldots, O_{T}^{1}, \ldots, O_{T}^{m}$ 
for an $m$-variate time series of length $T$ and independent 
realizations of such series. It is important to note that the latent 
Markov model is commonly associated with data of this type, but the 
estimation procedures that are used are not suitable for long time 
series due to underflow problems. Conversely the hidden Markov model 
is only used for univariate time series and multiple dependent time 
series or multivariate time series have not been considered in 
previous literature on hidden Markov models. In the next section, the 
likelihood and estimation procedure for this model is described, 
given data of the above form, for arbitrary $m$ and $T$. 


\section{Computing likelihood, gradients and Hessian}

Single-indicator latent class models. Estimation logic is different: 
the starting point is not a complete contingency table, which is then 
weeded. The starting point is a stochastic automaton which produces 
time series. In time series analysis, one does not expect a complete 
contingency table; in the contingency table logic, a single time 
series produces an enormous table with only a single filled cell. 

For computing the likelihood and its partial derivatives, I follow 
the strategy used by \cite{Lys2002}. They developed a method to 
compute exact gradients and information matrix by reformulating the 
forward procedure of the Baum-Welch  algorithm for estimating hidden 
Markov model parameters. Scores and information are computed at a 
single pass through the data, using the intermediate variables from 
the forward recursion. Their article only discusses univariate time 
series. Our aim in this section is to generalize their results to 
multiple dependent and independent time series. With the gradients 
available, it is possible to use general purpose optimization 
routines that allow box constraints and arbitrary (non-) linear 
constraints. 


\subsection{Likelihood}

The loglikelihood of hidden Markov models is usually computed by the 
so-called forward-backward algorithm \citep{Bau66,Rab89}, or rather 
by the forward part of this algorithm.
\cite{Lys2002} change the forward algorithm in such a way as to allow 
computing the gradients of the loglikelihood at the same time. They 
start by rewriting the likelihood as follows (for ease of exposition 
the dependence on the model parameters is dropped): 
\begin{equation}
	L_{T} = Pr(\vc{O}_{1}, \ldots, \vc{O}_{T}) = \prod_{t=1}^{T} 
Pr(\vc{O}_{t}|\vc{O}_{1}, 
	\ldots, \vc{O}_{t-1}), 
	\label{condLike}
\end{equation}
where $Pr(\vc{O}_{1}|\vc{O}_{0}):=Pr(\vc{O}_{1})$. Note that for a 
simple Markov chain these probabilities reduce to 
$Pr(\vc{O}_{t}|\vc{O}_{1},\ldots, 
\vc{O}_{t-1})=Pr(\vc{O}_{t}|\vc{O}_{t-1})$.
The log-likelihood can now be expressed as:
\begin{equation}
	l_{T} = \sum_{t=1}^{T} \log[Pr(\vc{O}_{t}|\vc{O}_{1}, \ldots, 
\vc{O}_{t-1})].
	\label{eq:condLogl}
\end{equation}

To compute the log-likelihood, \cite{Lys2002} define the following 
(forward) recursion:
\begin{align}
	\phi_{1}(j) &:= Pr(\vc{O}_{1}, S_{1}=j) = \pi_{j} b_{j}(\vc{O}_{1}) 
	\label{eq:fwd1} \\
\begin{split}
	\phi_{t}(j) &:= Pr(\vc{O}_{t}, S_{t}=j|\vc{O}_{1}, \ldots, 
\vc{O}_{t-1}) \\
	&= \sum_{i=1}^{N} [\phi_{t-1}(i)a_{ij}b_{j}(\vc{O}_{t})] \times 
(\Phi_{t-1})^{-1},
	\label{eq:fwdt} 
\end{split} 
\end{align}
where $\Phi_{t}=\sum_{i=1}^{N} \phi_{t}(i)$. Combining 
$\Phi_{t}=Pr(\vc{O}_{t}|\vc{O}_{1}, \ldots, \vc{O}_{t-1})$, and 
equation~(\ref{eq:condLogl}) gives the following expression for the 
log-likelihood:
\begin{equation}
	l_{T} = \sum_{t=1}^{T} \log \Phi_{t}.
\end{equation}

The above forward recursion can readily be generalized to mixture 
models, in which it is assumed that the data are realizations of a 
number of different LMMs and the goal is to assign posterior 
probabilities to sequences of observations. This situation occurs, 
for example, in learning data where different learning strategies may 
lead to different answer patterns. From an observed sequence of 
responses, it may not be immediately clear from which learning 
process they stem. Hence, it is interesting to consider a mixture of 
latent Markov models which incorporate restrictions that are 
consistent with each of the learning strategies. 

\subsection{Likelihood for mixtures}

The easiest way to compute the likelihood of a mixture of LMMs is by 
realizing that such a mixture is itself an LMM with a number of 
constraints on the initial parameters and the transition matrix. For 
example, consider a mixture of two LMMs, in which one component has 
two states and the other three states. Later in this paper I will 
present an example of such a mixture when  analyzing discrimination 
learning data. The  assumption is that children and adults use 
different strategies in such a task which can be translated into 
different latent Markov models. In this example the first component 
has two states and the following parameters: 
\begin{gather*}
\mat{A} =\begin{pmatrix} 
1 & 0\\ \alpha & 1-\alpha
\end{pmatrix} \quad \text{and} \quad
\pmb{\pi}=\begin{pmatrix} \pi_{1}\\ \pi_{2} \end{pmatrix}.
\end{gather*}
The response parameters $b$ are all to freely estimated. The second 
component has three  states and the following parameters:
\begin{gather*}
\mat{A} =\begin{pmatrix} 
1 & 0 & 0\\ 0 & g & 1-g \\ \alpha & \beta & \gamma
\end{pmatrix} \quad \text{and} \quad
\pmb{\pi}=\begin{pmatrix} \pi_{1}\\\pi_{2}\\\pi_{3}  \end{pmatrix}.
\end{gather*}
By combing the transition matrices and the initial vectors the 
following model results:
\begin{gather*}
\mat{A} =\begin{pmatrix} 
1 & 0 & 0 & 0 & 0\\
\alpha & 1-\alpha & 0 & 0 & 0 \\
0 & 0 & 1 & 0 & 0
\\0 & 0 & 0 & g & 1-g 
\\ 0 & 0 & \alpha & \beta & \gamma
\end{pmatrix} \quad \text{and} \quad
\pmb{\pi}=\begin{pmatrix} \pi_{1}^{1}\\ \pi_{2}^{1}  \\ 
\pi_{1}^{2}\\\pi_{2}^{2}\\\pi_{3}^{2} \end{pmatrix},
\end{gather*}
whith the constraint that the $\pi$'s sum to unity. It can easily be 
seen that computing the log-likelihood of such a model leads to many 
unneccessary computations because the transition matrix $\mat{A}$ 
contains blocks of zeroes that have to be multiplied $T$ times for 
each time series under consideration. Therefor, below the 
log-likelihood is computed as a real mixture of the components above 
with mixture parameters $p_{k}$. 

To compute the likelihood of a mixture of $K$ models, define the 
forward recursion variables as follows (these variables now have an 
extra index $k$ indicating that observation and transition 
probabilities are from latent model $k$):
\begin{align}
\begin{split}
\phi_{1}(j_{k}) &=  Pr(\vc{O}_{1}, 
S_{1}=j_{k})=p_{k}\pi_{j_{k}}b_{j_{k}}(\vc{O}_{1}).
\end{split}\label{eq:fwd1mix} \\
\begin{split}
\phi_{t}(j_{k})   &=   Pr(\vc{O}_{t}, S_{t}=j_{k}|\vc{O}_{1}, \ldots, 
\vc{O}_{t-1}) \\
			&= \left[ \sum_{k=1}^{K} \sum_{i=1}^{n_{k}} \phi_{t-1}(i_{k}) 
			a_{ij_{k}}b_{j_{k}}(\vc{O}_{t}) \right] \times (\Phi_{t-1})^{-1},
\end{split}\label{eq:fwdtmix} 
\end{align}
where $\Phi_{t} =  \sum_{k=1}^{K}\sum_{i=1}^{n_{k}} 
\phi_{t}(j_{k})$.  Note that the double sum over $k$ and $n_{k}$ is 
simply an enumeration of all the states of the model. Now, because 
$a_{ij_{k}}=0$ whenever $S_{i}$ is not part of component $k$, the sum 
over $k$ can be dropped and hence equation~\ref{eq:fwdtmix} reduces 
to:
\begin{equation}
	\phi_{t}(j_{k}) = \left[ \sum_{i=1}^{n_{k}} \phi_{t-1}(i_{k}) 
			a_{ij_{k}}b_{j_{k}}(\vc{O}_{t}) \right] \times (\Phi_{t-1})^{-1}
\end{equation}

Writing this out in matrices and vectors and rearranging provides 
useful expressions for implementational  purposes, for each $k$ we 
get:
\begin{align}
	\vc{\phi}_{1} &=  p_{k} \times \pmb{\pi} \otimes 
\vc{b}(\vc{O}_{1}),\\
	\vc{\phi}_{t} &= \left[ \mat{A}^{t} \times \pmb{\phi}_{t-1} \right ] 
		\otimes \vc{b} (\vc{O}_{t}) \times (\Phi_{t-1})^{-1},
\end{align}
where $^{t}$ denotes matrix tranpose, $\times$ denotes matrix/vector 
multiplication and $\otimes$ denotes the Hadamard product of 
vectors/matrices. 

\paragraph{Computational considerations} From 
equations~(\ref{eq:fwd1}--\ref{eq:fwdt}), it can be seen that 
computing the forward variables, and hence the log-likelihood, takes 
$O(Tn^{2})$ computations, for an $n$-state model and a time series of 
length $T$. Consider the mixture of two components, one with two 
states and the other with three states, as in the example above. 
Using equations~(\ref{eq:fwd1}--\ref{eq:fwdt}) to compute the 
log-likelihood of this model one needs $O(Tn^{2})=O(T\times 25)$ 
computations whereas with the mixture 
equationsequations~(\ref{eq:fwd1mix}--\ref{eq:fwdtmix}), one only 
needs $\sum_{n_{i}} O(n_{i}^{2}T)$ computations, in this case $O(T 
\times 13)$. So, it can  be seen that in this easy example the 
computational cost is almost halved.  


Note that in above equations $b_{j}(\vc{O}_{t})$ factors into 
$\prod_{i=1}^{m} b_{j} (O_{t}^{i})$, as is the case in the latent 
class model. 

%\subsection{Posterior probabilities}

%Viterbi algorithm?

%Using the equations above for the likelihood of single and mixture 
models, arriving at the posterior probabilities of membership of the 
components can be easily accomplished. 

%Nou, nee dus!

%The posterior probability is the probability that data $\vc{O}$ are 
generated by mixture component $k$, i.e., $Pr(C=k|\vc{O})$, where $C$ 
denotes the mixture component. This probability can be expressed as:
%\begin{eqnarray}
%	Pr(C=k|\vc{O}) &=& \frac{Pr(C=k) Pr(\vc{O}|C=k)}
%	{\sum_{k=1}^{K} Pr(C=k)Pr(\vc{O}|C=k)} \\
%	&=& \frac {p_{k} l_{TMC=k}}
%	{\sum}
%\end{eqnarray}


\subsection{Gradients}

\newcommand{\fpp}{\frac{\partial} {\partial \lambda_{1}}}

See equations 10--12  in \cite{Lys2002} for the score recursion 
functions of the hidden Markov model for a univariate time series. 
Here the corresponding score recursion for the multivariate mixture 
case are provided. The $t=1$ components of this score recursion are 
defined as (for an arbitrary parameter $\lambda_{1}$):
\begin{align}
\psi_{1}(j_{k};\lambda_{1}) &:=  \fpp Pr(\vc{O}_{1}|S_{1}=j_{k}) \\
\begin{split} 
	&= \left[  \fpp p_{k} \right] \pi_{j_{k}}b_{j_{k}}(\vc{O}_{1}) + 
	p_{k}\left[ \fpp \pi_{j_{k}} \right] b_{j_{k}}(\vc{O}_{1}) \\
	& \qquad  + p_{k}\pi_{j_{k}} \left[ \fpp 
b_{j_{k}}(\vc{O}_{1})\right],
\end{split} \label{eq:psi1}
\end{align}
and for $t>1$ the definition is:
\begin{align}
\psi_{t}(j_{k};\lambda_{1})  & =  \frac{\fpp Pr(\vc{O}_{1}, \ldots, 
\vc{O}_{t}, S_{t}=j_{k})}
			{Pr(\vc{O}_{1}, \ldots, \vc{O}_{t-1})}  \\
\begin{split} 
			 & =  %\sum_{k=1}^{K} klopt dit?? is \fpp a_{ij} wel altijd 0 voor andere componenten??
			 \sum_{i=1}^{n_{k}} \Bigg\{ \psi_{t-1}(i;\lambda_{1})a_{ij_{k}} 
			 b_{j_{k}}(\vc{O}_{t}) \\ 
			 &\qquad +\phi_{t-1}(i) \left[ \fpp a_{ij_{k}} \right] b_{j_{k}} 
(\vc{O}_{t}) \\
			&\qquad +\phi_{t-1}(i)a_{ij_{k}}  \left[ \fpp b_{j_{k}} 
(\vc{O}_{t}) \right] \Bigg\} 
			\times (\Phi_{t-1})^{-1}.
\end{split} \label{eq:psit}
\end{align}

Writing this out in matrix notation (and dropping the $k$-subscripts 
for components) we get for $t=1$ and for each $k$:
\begin{eqnarray}
	\pmb{\psi}_{1}(\lambda_{1}) = \vc{\fpp} p_{k} \times \pmb{\pi} 
\otimes \vc{b}(O_{1})+
		p_{k} \times \vc{\fpp}\pmb{\pi} \otimes \vc{b}(O_{1}) +
		p_{k} \times \pmb{\pi} \otimes \fpp \vc{b}(O_{1}),
\end{eqnarray}
and for $\pmb{\psi}_{t}$ we get: 
\begin{equation}
\begin{split} 
	\pmb{\psi}_{t}(\lambda_{1}) 
	&= \left[ \mat{A}^{t} \times \pmb{\psi_{t-1}}(\lambda_{1})  \right ] 
\otimes \vc{b}(O_{t}) +
	\left [ \left[ \fpp \mat{A} \right ]^{t} \times \pmb{\phi}_{t-1} 
\right ] \otimes \vc{b}(O_{t}) \\
	& \qquad + \left [ \mat{A}^{t} \times \pmb{\phi}_{t-1}  \right ] 
\otimes \fpp \vc{b}(O_{t}).
\end{split} 
\end{equation}

Using above equations, \cite{Lys2002} derive the following equation 
for the partial derivative of the likelihood:
\begin{equation}
	\fpp l_{T}= 	
		\frac{\mathbf{\Psi}_{T}(\lambda_{1})}{\mathbf{\Phi}_{T}},
\end{equation}
where $\Psi_{t}=\sum_{k=1}^{K} \sum_{i=1}^{n_{k}} 
\psi_{t}(j_{k};\lambda_{1})$. 
Starting from the equation from the logarithm of the likelihood, this 
is easily seen to be correct: 
\begin{eqnarray}
	\fpp \log Pr(\vc{O}_{1}, \ldots, \vc{O}_{T}) &=& Pr(\vc{O}_{1}, 
\ldots, \vc{O}_{T})^{-1} 
	\fpp Pr(\vc{O}_{1}, \ldots, \vc{O}_{T}) \\
	&=& Pr(\vc{O}_{1}, \ldots, \vc{O}_{T})^{-1} \Psi_{T}(\lambda_{1}) 
Pr(\vc{O}_{1}, \ldots, \vc{O}_{T-1}) \\
	&=&  \frac{Pr(\vc{O}_{1}, \ldots, \vc{O}_{T-1})}{Pr(\vc{O}_{1}, 
\ldots, \vc{O}_{T})}  \Psi_{T} (\lambda_{1}) \\
	&=&  \frac{\mathbf{\Psi}_{T}(\lambda_{1})}{\mathbf{\Phi}_{T}}.
\end{eqnarray}


Having provided the global computations above we now need expressions 
for $\fpp p_{k}$, $\fpp \pi_{i}$, $\fpp a_{ij}$, and $\fpp 
b_{i}(\vc{O}_{t})$, which are dependent on the particular 
$\lambda_{1}$ under consideration. This generates   special cases for 
each combination of $\lambda_{1}$ and each of the components of the 
gradient computation. These cases are worked out below for each of 
the choices $\lambda_{1} \in \pmb{\lambda}=\{ 
\vc{p},\pmb{\pi},\mat{A},\vc{b} \}$.


%%p-k cases
\paragraph{Case 1: $\fpp p_{k}$}

\begin{align} 
        \fpp p_{k} = \left\{ \begin{array}{ll} 
            1 & \mbox{if $\lambda_{1}=p_{k}$} \\ 
            0 & \mbox{otherwise} 
            \end{array} \right. 
\end{align} 

%%pipi-k cases

\paragraph{Case 2: $\fpp \pi_{i}$ }

\begin{align} 
        \fpp \pi_{i}^{k} = \left\{ \begin{array}{ll} 
            1 & \mbox{if $\lambda_{1}=\pi_{i}^{k}$} \\ 
            0 & \mbox{otherwise} 
            \end{array} \right. 
\end{align} 

\paragraph{Case 3: $\fpp a_{ij}$ }

\begin{align} 
        \fpp a_{ij}^{k}= \left\{ \begin{array}{ll} 
            1 & \mbox{if $\lambda_{1}=a_{ij}^{k}$} \\ 
            0 & \mbox{otherwise} 
            \end{array} \right. 
\end{align} 

The cases below for $\fpp b_{j}(\vc{O}_{t})$ are slightly more 
interesting. First, remember that in the case of an $m$-variate time 
series of observations, $\fpp b_{j}(\vc{O}_{t})=\fpp b_{j}(O_{t}^{1}, 
\ldots, O_{t}^{m})$, of which we need partial derivatives for 
arbitrary parameters $\lambda_{1}$. First note that because of local 
independence we can write:
\begin{equation}
	\fpp [b_{j}(O_{t}^{1}, \ldots, O_{t}^{m})]= \frac{\partial} 
{\partial \lambda_{1} } [b_{j}(O_{t}^{1})] \times  
[b_{j}(O_{t}^{2})], \ldots,  [b_{j}(O_{t}^{m})].  
\end{equation}
Applying the chain rule for products we get:
\begin{equation}
	\fpp [b_{j}(O_{t}^{1}, \ldots, O_{t}^{m})] =
	\sum_{l=1}^{m} \left[ \prod_{i=1, \ldots, \hat{l}, \ldots, m} 
b_{j}(O_{t}^{i}) \right] \times
	\fpp  [b_{j}(O_{t}^{l})],
	\label{partialProd}
\end{equation}
where $\hat{l}$ means that that term is left out of the product. 
These latter terms, $\frac{\partial} {\partial \lambda_{1} }  
[b_{j}(O_{t}^{k})]$, are easy to compute given either multinomial or 
gaussian observation densities. As a consequence, this provides 
gradients for the multivariate time series model for mixed 
categorical and gaussian observations, e.g.\ correct/incorrect 
responses and reaction times, and a categorical latent state or 
class.  Hence, we need the partial derivatives $\fpp 
b_{j}({O}_{t}^{l})$ instead of $\fpp b_{j}(\vc{O}_{t})$ (if 
$\lambda_{1} \in \{ \vc{p}, \pmb{\pi},\mat{A} \} $, then $\fpp 
b_{j}=0$).


\paragraph{Case 4: $\fpp b_{j}({O}_{t}^{l})$ and $\lambda_{1} \in 
\vc{b}$.}  These derivatives depend on the response function of the 
item, i.e.\ whether the item is multinomial or gaussian (other 
itemtypes are easy to add). For the derivatives of the response 
functions there are two cases depending on the type of the response 
function. For a gaussian item we have the following:
\begin{align} 
        \fpp b_{j_{k}}(O_{t}^{l})= \left\{ 
	\begin{array}{ll} 
		b_{j_{k}}(O_{t}^{l}) \times  \left[  \frac{O_{t}-\mu}{\sigma^{2}}  
\right] & 
			\mbox{if $\lambda_{1} = \mu_{j_{k}}^{l} $ }  \\ 
            	b_{j_{k}}(O_{t}^{l}) \times \left[  
\frac{(O_{t}-\mu)^{2}}{\sigma^{3}} - \frac{1}{\sigma} \right] & 
			\mbox{if $\lambda_{1} = \sigma_{j_{k}}^{l} $} \\
		0 & \mbox{otherwise} 
            \end{array} \right. 
\end{align}
		
For an $c$-category item $l$ with response probabilities $p_{1}, 
\ldots, p_{c}$ we have the following:
\begin{align} 
        \fpp b_{j_{k}}(O_{t}^{l})= \left\{ 
	\begin{array}{ll} 
		1 & \mbox{if $\lambda_{1} = p_{i}$ \& $O_{t}=i$}  \\ 
		0 & \mbox{otherwise} 
            \end{array} \right. 
\end{align}


\subsection{Observed information}

\newcommand{\fpt}{\frac{\partial}{\partial \lambda_{2}} }

\newcommand{\fps}{\frac{\partial^{2}}{\partial\lambda_{1}\partial\lambda_{2}}}



To compute the observed information, again the formulae by 
\citet{Lys2002} are applied and generalized to the m-variate mixture 
case. Define the information recursion as follows:
\begin{align}
\begin{split}
	\omega_{1}(j_{k};\lambda_{1},\lambda_{2}) 
	=  & \, \fps Pr(\vc{O}_{1}, S_{1}=j_{k}) = \fps p_{k} \pi_{j_{k}} 
b_{j_{k}}(\vc{O}_{1})
	\\  = & \, \fpp \left[ \fpt p_{k} \pi_{j_{k}} b_{j_{k}}(\vc{O}_{1}) 
\right] 
	\\  =  & \fpp  \bigg\{  \left[ \fpt p_{k} \right]  \pi_{j_{k}} 
b_{j_{k}}(\vc{O}_{1}) +
		\, p_{k} \left[ \fpt \pi_{j_{k}} \right]  b_{j_{k}}(\vc{O}_{1}) + 
	\\  &  \, p_{k} \pi_{j_{k}} \left[ \fpt  b_{j_{k}}(\vc{O}_{1}) 
\right] \bigg\}
	\\  =  & \,  \left[ \fps p_{k} \right] \pi_{j_{k}} 
b_{j_{k}}(\vc{O}_{1}) + 
		\left[ \fpt p_{k} \right] \cdot \left[\fpp \pi_{j_{k}} \right] 
b_{j_{k}}(\vc{O}_{1}) +
	\\  &   \, \left[ \fpt p_{k} \right] \pi_{k} \left[ \fpp b_{j_{k}} 
(\vc{O}_{1}) \right] +
		\left[ \fpp p_{k} \right] \pi_{j_{k}} b_{j_{k}}(\vc{O}_{1}) + 
	\\ &   \, p_{k} \left[ \fps \pi_{j_{k}} \right] 
b_{j_{k}}(\vc{O}_{1}) +
		p_{k} \left[ \fpt \pi_{j_{k}} \right] \cdot \left[ \fpp 
b_{j_{k}}(\vc{O}_{1}) \right] +
	\\ &   \, \left[ \fpp p_{k} \right] \pi_{j_{k}} \left[ \fpt 
b_{j_{k}}(\vc{O}_{1}) \right] +
		p_{k} \left[ \fpp \pi_{j_{k}} \right] \cdot \left[ \fpt 
b_{j_{k}}(\vc{O}_{1}) \right] +
	\\ &   \, p_{k} \pi_{j_{k}} \left[ \fps b_{j_{k}}(\vc{O}_{1}) 
\right].
\end{split}
\end{align}

Similarly, for $t>1$, the information recursion is defined as:
\begin{align}
\begin{split}
	\omega_{t}(j_{k};\lambda_{1},\lambda_{2}) =  
	&\frac{\fps Pr(\vc{O}_{1}, \ldots, 
\vc{O}_{t},S_{t}=j_{k})}{Pr(\vc{O}_{1}, \ldots, \vc{O}_{t-1})}
	\\ = &\sum_{k=1}^{K} \sum_{i=1}^{n_{k}} \Bigg\{ 
	\omega_{t-1}(i;\lambda_{1},\lambda_{2}) a_{ij_{k}} 
b_{j_{k}}(\vc{O}_{t}) +
	\psi_{t-1}(i;\lambda_{1}) \left[ \fpt a_{ij_{k}} \right] 
b_{j_{k}}(\vc{O}_{t}) +
	\\  &\psi_{t-1}(i;\lambda_{1})  a_{ij_{k}}  \left[ \fpt 
b_{j_{k}}(\vc{O}_{t}) \right] +
	\psi_{t-1}(i;\lambda_{2}) \left[ \fpp a_{ij_{k}} \right] 
b_{j_{k}}(\vc{O}_{t}) + 
	\\ & \psi_{t-1}(i;\lambda_{2}) a_{ij_{k}} \left[ \fpp 
b_{j_{k}}(\vc{O}_{t})\right]  + 
	\phi_{t-1}(i) \left[ \fps a_{ij_{k}} \right]  b_{j_{k}}(\vc{O}_{t}) +
	\\ & \phi_{t-1}(i) a_{ij_{k}} \left[ \fps b_{j_{k}}(\vc{O}_{t}) 
\right]  +
	\phi_{t-1}(i) \left[ \fpp a_{ij_{k}} \right] \left[ \fpt 
b_{j_{k}}(\vc{O}_{t}) \right]  +
	\\ & \phi_{t-1}(i) \left[ \fpt a_{ij_{k}}\right] \left[ \fpp 
b_{j_{k}}(\vc{O}_{t}) \right]  \Bigg\} 
	\times \frac{1}{\Phi_{t-1}}
\end{split}
\end{align}

It is important here to provide the second order partial derivatives 
of the terms that are involved in the multi-variate case. These are 
the $b_{j_{k}}(\cdot)$ terms. Remember that these terms can be 
factored and hence the derivatives can be derived from repeatedly 
applying the chain rule for products and using equation 
\ref{partialProd} from above for the first order derivatives: 
\begin{align}
\begin{split}
	\fps b_{j_{k}} (\vc{O}_{t}) & = \fps \prod_{i=1}^{m} b_{j_{k}}^{i} 
(O_{t}^{i}) 
	\\ & = \fpp \left[ \fpt \prod_{i=1}^{m} b_{j_{k}}^{i} (O_{t}^{i})  
\right] 
	\\ & = \fpp \Bigg\{ \sum_{l=1}^{m} \left[ \prod_{i=1, \ldots, 
\hat{l}, \ldots, m} b_{j}(O_{t}^{i}) 
	\right] \times \fpt  [b_{j}(O_{t}^{l})] \Bigg\}  
	\\ & = \sum_{l=1}^{m} \left[ \fpp \prod_{i=1, \ldots, \hat{l}, 
\ldots, m} b_{j}(O_{t}^{i}) \right] 
	\times \fpt  \left[ b_{j}(O_{t}^{l}) \right] \; + 
	\\ &\phantom{=}\;\,  \sum_{l=1}^{m}  \left[ \prod_{i=1, \ldots, 
\hat{l}, \ldots, m} b_{j}(O_{t}^{i}) 
	\right] \times \fpp \fpt  \left[ b_{j}(O_{t}^{l})\right]  
	\\ & = \sum_{k=1,k\neq	l}^{m} \sum_{l=1}^{m}  \left[ \prod_{i=1, 
\ldots, \hat{l}, \hat{k}, \ldots, m}
	b_{j}(O_{t}^{i}) \right] \fpt  \left[ b_{j}(O_{t}^{k})\right]  \fpt  
\left[ b_{j}(O_{t}^{l})\right] \;+ 
	\\ &\phantom{=}\;\, \sum_{l=1}^{m}  \left[ \prod_{i=1, \ldots, 
\hat{l}, \ldots, m} b_{j}(O_{t}^{i}) 
	\right] \times \fps  \left[b_{j}(O_{t}^{l}) \right] 
\end{split}
\end{align}


In order to compute the above terms we need for arbitrary parameters 
$\lambda_{1}$ the terms $\fps  \lambda_{1}$. Fortunately, these are 
easy to compute as most of them are zero. In particular, with 
$\lambda_{1} \in \{\mat{A},\vc{p},\pmb{\pi} \}$, the double 
derivatives are zero and we only need them when 
$\lambda_{1}\in\{\mat{B}\}$.
These latter ones are again zero in the case of multinomial items and 
so only the gaussian observations remain which are given here in four 
cases:  

%%p-k cases
\paragraph{Case 1: $\fps b_{j}(O_{t}^{l}), \lambda_{1}=\mu, 
\lambda_{2}=\mu$}

\begin{align} 
        \fps b_{j_{k}}(O_{t}^{l})=b_{j}(O_{t}^{l}) \left[ \left[ 
\frac{O_{t}^{l}-\mu}{\sigma^{2}} \right]^{2} - \frac{1}{\sigma^{2}} 
\right]
\end{align}

\paragraph{Case 2: $\fps b_{j}(O_{t}^{l}), \lambda_{1}=\mu, 
\lambda_{2}=\sigma$}

\begin{align} 
        \fps b_{j_{k}}(O_{t}^{l})=b_{j}(O_{t}^{l}) \left[ \left[ 
\frac{(O_{t}^{l}-\mu)^{2}}{\sigma^{3}} -
	\frac{1}{\sigma}\right]  \cdot \left[ 
\frac{O_{t}^{l}-\mu}{\sigma^{2}} \right] + 
	\left[ \frac{2\mu-2O_{t}^{l}}{\sigma^{3}} \right] \right]
\end{align}


\paragraph{Case 3: $\fps b_{j}(O_{t}^{l}), \lambda_{1}=\sigma, 
\lambda_{2}=\mu$}

\begin{align} 
        \fps b_{j_{k}}(O_{t}^{l})=b_{j}(O_{t}^{l})  \left[ \left[ 
\frac{(O_{t}^{l}-\mu)^{2}}{\sigma^{3}} -
	\frac{1}{\sigma}\right]  \cdot \left[ 
\frac{O_{t}^{l}-\mu}{\sigma^{2}} \right] + 
	\left[ \frac{2\mu-2O_{t}^{l}}{\sigma^{3}} \right] \right]
\end{align}

\paragraph{Case 4: $\fps b_{j}(O_{t}^{l}), \lambda_{1}=\sigma, 
\lambda_{2}=\sigma$}

\begin{align} 
        \fps b_{j_{k}}(O_{t}^{l})=b_{j}(O_{t}^{l}) \left[ \left[ 
\frac{(O_{t}^{l}-\mu)^{2}}{\sigma^{3}} -
	\frac{1}{\sigma}\right]^{2} + \left[  \frac{1}{\sigma^{2}} - 
\frac{3(O_{t}^{l}-\mu)^{2}}{\sigma^{4}}  \right]  \right]
\end{align}

		
\bibliography{psy,ingmar,hmm}

% \section*{Author note}

\end{document}





