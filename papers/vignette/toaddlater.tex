
\subsection{Mixtures of LMMs}

The above forward recursion can readily be generalized to mixture 
models, in which it is assumed that the data are realizations of a 
number of different LMMs and the goal is to assign posterior 
probabilities to sequences of observations. This situation occurs, 
for example, in learning data where different learning strategies may 
lead to different answer patterns. From an observed sequence of 
responses, it may not be immediately clear from which learning 
process they stem. Hence, it is interesting to consider a mixture of 
latent Markov models which incorporate restrictions that are 
consistent with each of the learning strategies. 

To compute the likelihood of a mixture of $K$ models, define the 
forward recursion variables as follows (these variables now have an 
extra index $k$ indicating that observation and transition 
probabilities are from latent model $k$):
\begin{align}
\begin{split}
\phi_{1}(j_{k}) &=  Pr(\vc{O}_{1}, 
S_{1}=j_{k})=p_{k}\pi_{j_{k}}b_{j_{k}}(\vc{O}_{1}).
\end{split}\label{eq:fwd1mix} \\
\begin{split}
\phi_{t}(j_{k})   &=   Pr(\vc{O}_{t}, S_{t}=j_{k}|\vc{O}_{1}, \ldots, 
\vc{O}_{t-1}) \\
			&= \left[ \sum_{k=1}^{K} \sum_{i=1}^{n_{k}} \phi_{t-1}(i_{k}) 
			a_{ij_{k}}b_{j_{k}}(\vc{O}_{t}) \right] \times (\Phi_{t-1})^{-1},
\end{split}\label{eq:fwdtmix} 
\end{align}
where $\Phi_{t} = \sum_{k=1}^{K}\sum_{i=1}^{n_{k}} \phi_{t}(j_{k})$.
Note that the double sum over $k$ and $n_{k}$ is simply an enumeration
of all the states of the model.  Now, because $a_{ij_{k}}=0$ whenever
$S_{i}$ is not part of component $k$, the sum over $k$ can be dropped
and hence equation~\ref{eq:fwdtmix} reduces to:
\begin{equation}
	\phi_{t}(j_{k}) = \left[ \sum_{i=1}^{n_{k}} \phi_{t-1}(i_{k}) 
			a_{ij_{k}}b_{j_{k}}(\vc{O}_{t}) \right] \times (\Phi_{t-1})^{-1}
\end{equation}
The loglikelihood is computed by applying equation~\ref{eq:logl} on
these terms.  For multiple cases, the log-likelihood is simply the sum
over the individual log-likelihoods. 


Consider a mixture of two components, one with two states
and the other with three states.  Using
equations~(\ref{eq:fwd1}--\ref{eq:fwdt}) to compute the log-likelihood
of this model one needs $O(Tn^{2})=O(T\times 25)$ computations whereas
with the mixture equations~(\ref{eq:fwd1mix}--\ref{eq:fwdtmix}),
$\sum_{n_{i}} O(n_{i}^{2}T)$ computations are needed, in this case
$O(T \times 13)$.  So, it can be seen that in this easy example the
computational cost is almost halved.

\section{Gradients}

\newcommand{\fpp}{\frac{\partial} {\partial \lambda_{1}}}

See equations 10--12  in \cite{Lystig2002} for the score recursion 
functions of the hidden Markov model for a univariate time series. 
Here the corresponding score recursion for the multivariate mixture 
case are provided. The $t=1$ components of this score recursion are 
defined as (for an arbitrary parameter $\lambda_{1}$):
\begin{align}
\psi_{1}(j_{k};\lambda_{1}) &:=  \fpp Pr(\vc{O}_{1}|S_{1}=j_{k}) \\
\begin{split} 
	&= \left[  \fpp p_{k} \right] \pi_{j_{k}}b_{j_{k}}(\vc{O}_{1}) + 
	p_{k}\left[ \fpp \pi_{j_{k}} \right] b_{j_{k}}(\vc{O}_{1}) \\
	& \qquad  + p_{k}\pi_{j_{k}} \left[ \fpp 
b_{j_{k}}(\vc{O}_{1})\right],
\end{split} \label{eq:psi1}
\end{align}
and for $t>1$ the definition is:
\begin{align}
\psi_{t}(j_{k};\lambda_{1})  & =  \frac{\fpp Pr(\vc{O}_{1}, \ldots, 
\vc{O}_{t}, S_{t}=j_{k})}
			{Pr(\vc{O}_{1}, \ldots, \vc{O}_{t-1})}  \\
\begin{split} 
	& =  
			 \sum_{i=1}^{n_{k}} \Bigg\{ \psi_{t-1}(i;\lambda_{1})a_{ij_{k}} 
			 b_{j_{k}}(\vc{O}_{t}) \\ 
			 &\qquad +\phi_{t-1}(i) \left[ \fpp a_{ij_{k}} \right] b_{j_{k}} 
(\vc{O}_{t}) \\
			&\qquad +\phi_{t-1}(i)a_{ij_{k}}  \left[ \fpp b_{j_{k}} 
(\vc{O}_{t}) \right] \Bigg\} 
			\times (\Phi_{t-1})^{-1}.
\end{split} \label{eq:psit}
\end{align}

Using above equations, \cite{Lystig2002} derive the following equation 
for the partial derivative of the likelihood:
\begin{equation}
	\fpp l_{T}= 	
		\frac{\mathbf{\Psi}_{T}(\lambda_{1})}{\mathbf{\Phi}_{T}},
\end{equation}
where $\Psi_{t}=\sum_{k=1}^{K} \sum_{i=1}^{n_{k}} 
\psi_{t}(j_{k};\lambda_{1})$. 
Starting from the equation from the logarithm of the likelihood, this 
is easily seen to be correct: 
\begin{eqnarray*}
	\fpp \log Pr(\vc{O}_{1}, \ldots, \vc{O}_{T}) &=& Pr(\vc{O}_{1}, 
\ldots, \vc{O}_{T})^{-1} 
	\fpp Pr(\vc{O}_{1}, \ldots, \vc{O}_{T}) \\
	&=&  \frac{Pr(\vc{O}_{1}, \ldots, \vc{O}_{T-1})}{Pr(\vc{O}_{1}, 
\ldots, \vc{O}_{T})}  \Psi_{T} (\lambda_{1}) \\
	&=&  \frac{\mathbf{\Psi}_{T}(\lambda_{1})}{\mathbf{\Phi}_{T}}.
\end{eqnarray*}

Further, to actually compute the gradients, the partial derivatives of
the parameters and observation distribution functions are neccessary,
i.e., $\fpp p_{k}$, $\fpp \pi_{i}$, $\fpp a_{ij}$, and $\fpp
\vc{b}_{i}(\vc{O}_{t})$.  Only the latter case requires some
attention.  We need the following derivatives $\fpp
\vc{b}_{j}(\vc{O}_{t})=\fpp \vc{b}_{j}(O_{t}^{1}, \ldots, O_{t}^{m})$, for
arbitrary parameters $\lambda_{1}$. To stress that $\vc{b}_{j}$ is a
vector of functions, we here used boldface. First note that because of local
independence we can write:
\begin{equation*}
	\fpp \left[ b_{j}(O_{t}^{1}, \ldots, O_{t}^{m}) \right] = \frac{\partial} 
{\partial \lambda_{1} } \left[ b_{j}(O_{t}^{1}) \right] \times  
\left[ b_{j}(O_{t}^{2}) \right], \ldots,  \left[ b_{j}(O_{t}^{m}) \right].  
\end{equation*}
Applying the chain rule for products we get:
\begin{equation}
	\fpp [b_{j}(O_{t}^{1}, \ldots, O_{t}^{m})] =
	\sum_{l=1}^{m} \left[ \prod_{i=1, \ldots, \hat{l}, \ldots, m} 
b_{j}(O_{t}^{i}) \right] \times
	\fpp  [b_{j}(O_{t}^{l})],
	\label{partialProd}
\end{equation}
where $\hat{l}$ means that that term is left out of the product. 
These latter terms, $\frac{\partial} {\partial \lambda_{1} }  
[b_{j}(O_{t}^{k})]$, are easy to compute given either multinomial or 
gaussian observation densities $b_{j}(\dot)$

\subsection{Generating data}

The \code{dmm}-class has a \code{generate} method that can be used to 
generate data according to a specified model. 

\begin{verbatim}
gen<-generate(c(100,50),mod)
\end{verbatim}


\section{Multi group/case analysis}

\begin{verbatim}
conpat=rep(1,15)
conpat[1]=0
conpat[8:9]=0
conpat[14:15]=0
stv=c(1,0.9,0.1,0.1,0.9,5.5,0.2,0.5,0.5,6.4,0.25,0.9,0.1,0.5,0.5)
mod<-dmm(nstates=2,itemt=c("n",2),stval=stv,conpat=conpat)
\end{verbatim}

\code{depmix4} can handle multiple cases or multiple groups. A
multigroup model is specified using the function \code{mgdmm} as
follows:

\begin{verbatim}
mgr <- mgdmm(dmm=mod,ng=3,trans=TRUE,obser=FALSE)
mgrfree <- mgdmm(dmm=mod,ng=3,trans=FALSE)
\end{verbatim}

The \code{ng} argument specifies the number of groups, and the
\code{dmm} argument specifies the model for each group.  \code{dmm}
can be either a single model or list of models oflength \code{ng}.  If
it is a single model, each group has an identical structural model
(same fixed and constrained parameters).  Three further arguments can
be used to constrain parameters between groups, \code{trans},
\code{obser}, and \code{init} respectively.  By setting either of
these to \code{TRUE}, the corresponding transition, observation, and
initial state parameters are estimated equal between
groups\footnote{There is at this moment no way of fine-tuning this to
restrict equalities to individual parameters.  However, this can be
accomplished by manually changing the linear constraint matrix, and
the corresponding upper and lower boundaries.}.

In this example, the model from above was used and fitted on the three
observed series, and the \code{trans=TRUE} ensures that the transition
matrix parameters are constrained to be equal between the models for
these series, whereas the observation parameters are freely estimated,
i.e.\ to capture learning effects. 

The loglikelihood ratio statistic can be used to test whether
constraining these transition parameters significantly reduces the
goodness-of-fit of the model.  The statistic has an approximate
$\chi^{2}$ distribution with $df=4$ because in each but the first
model, two transition matrix parameters were estimated equal to the
parameters in the first model (note that the other two transition
parameters already had to be constrained to ensure that the rows of
the transition matrices sum to 1).


\section{Mixtures of latent Markov models}

\code{depmix4} provides support for fitting mixtures of latent Markov
models using the \code{mixdmm} function; it takes a list of
\code{dmm}'s as argument, possibly together with the starting values
for the mixing proportions for each component model.  There's an
example in the helpfiles.  It fits the model to data from a
discrimination learning experiment which is provided as data set
\code{discrimination} \cite{Raijmakers2001}. 

\section{Finite mixtures and latent class models}

The function \code{lca} can be used to specify latent class models
and/or finite mixture models.  It is simply a wrapper for the
\code{dmm} function, and all it does is adding appropriate numbers of
zeroes and ones to the parameter specification vectors for starting
values, fixed values and linear constraints.  When a model has class
\code{lca} the summary function does not print the transition matrix
(because it is fixed and/or not estimated).


\section{Starting values}

Although providing your own starting values is preferable,
\pkg{depmixS4} has a routine for generating starting values using the
\code{kmeans}-function from the \pkg{stats}-package.  This will
usually provide reasonable starting values, but can be way off in a
number of cases.  First, for univariate categorical time series,
\code{kmeans} does not work at all, and \pkg{depmixS4} will provide a
warning.  Second, for multivariate series with unordered categorical
items with more than 2 categories, \code{kmeans} may provide good
starting values, but they may similarly be completely off, due to the
implicit assumption in \code{kmeans} that the categories are
indicating an underlying continuum.  Starting values using
\code{kmeans} are automatically provided when a model is specified
without starting values.  The argument \code{kmst} to the
\code{fitdmm}-function can be used to control this behavior.

Starting values of the parameters, either user provided or generated,
can be further boosted by using posterior estimates using the Viterbi
algorithm \cite{Rabiner1989}.  That is, first the a posteriori latent
states are generated from the current parameter values for the data at
hand.  Next, from the a posteriori latent states, new parameter
estimates are derived.  This is done by default and can be controlled
by the \code{postst} argument.  Provided that the starting values were
close to their true values, using this procedure further pushes those
parameters in the right direction.  If however the original values
were bad, this procedure may result in bad estimates, i.e.,
optimization will lead to some non-optimal local maximum of the
loglikelihood.



